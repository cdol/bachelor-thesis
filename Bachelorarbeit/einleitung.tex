\section*{Einleitung}
Röntgenmikroskopie ist ein Bildgebungsverfahren, welches ein Auflösungsvermögen zwischen einem Lichtmikroskop und SEM (\textit{scanning electron microscope}) besitzt. Dabei kann das Transmissionssignal mittels STXM (\textit{scanning transmission X-ray microscopy}) detektiert werden. Außerdem ist es möglich den Aufbau um einen optionalen Fluoreszenzdetektor zu erweitern, wie im Fall des AnImaX-Projekts (\textit{Analytical Imaging with X-Rays}) der AG Kanngießer (TU Berlin). Das Projekt beschäftigt sich im Themengebiet der Röntgenfluoreszenz an einem Aufbau, welcher gleichzeitige Detektion von Fluoreszenzsignalen, sowie die dazugehörigen Transmissionssignalen erlaubt. Dafür wurde eigens ein QUAD Silizium Drift Detektor (SDD) von \textit{Bruker Nano} angefertigt, welcher im AnImaX Röntgenmikroskop verbaut ist. Der Detektor ermöglicht die Aufnahme eines großen Raumwinkels ($>\SI{1}{\steradian}$), daher ist er beispielsweise für biologische Proben, welche nur ein schwaches Fluoreszenzsignal emittieren, geeignet. \newline
Durch den großen Raumwinkel bedingt sind herkömmliche Quantifizierungsansätze für diesen Detektor nicht mehr gültig. Die Fluoreszenzstrahlung kann nicht mehr als parallel angenommen werden, da sie verschiedene Pfade durch die Probe nimmt. Je nach Aufbau der Probe variieren daher sowohl die Weglänge durch die Probe, als auch die Absorptionskoeffizienten aufgrund wechselnder chemischer Kompositionen. \newline
Als Teil der Arbeit \textit{Simulation of Solid Angle Effects in the Detection of X-ray Fluorescence upon Investigation of Inhomogeneous Samples} von Hanna Dierks ist dadurch das Simulationsprogramm QUADaps entstanden, welches diese Effekte aufgrund einer virtuellen Beschreibung des QUAD Detektors simuliert und Grundsteine für eine Quantifizierung in der Röntgenfluoreszenz unter großen Raumwinkeln legen soll. \newline
In dieser Arbeit soll nun an erste Ergebnisse angeknüpft werden. Das Hauptaugenmerk liegt auf dem Entwurf von Proben, welche in den nachfolgenden Kapiteln logisch aufgebaut werden. Diese sollen simuliert und im Anschluss an diese Arbeit experimentell untersucht werden. Der Vergleich von simulierten und experimentell ermittelten Daten soll zur Validierung des Simulationsprogramms QUADaps beitragen.

