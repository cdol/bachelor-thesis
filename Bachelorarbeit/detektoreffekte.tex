\section{Detektor- und Streuungseffekte}
Nachdem im vorangestellten Abschnitt auf das Wirkungsprinzip eines SDD eingegangen wurde, soll in diesem Abschnitt auf beobachtbare physikalische Effekte eingegangen werden, welche als Messartefakte in den detektierten Daten hinterbleiben. Diese können beispielsweise durch die Art des Detektors, allgemein aber auch durch diverse Streuprozesse wie elastische (Rayleigh) oder inelastische Streuung (Compton) bedingt sein. Ignorieren dieser Effekte würde zwangsweise zu falsch interpretierten Daten führen, wie am folgenden Beispiel ersichtlich:\newlines
Erreichen 2 Photonen einer monochromatischen Quelle die aktive Detektorfläche zwischen zwei Ausleseperioden, so sind die durch die Photonen ausgelösten Ladungsträger\-wolken für die Ausleseelektronik voneinander ununterscheidbar. Dadurch wird die voll\-ständige Ladung einem Photon mit der doppelten Photonenenergie zugeordnet, Pile-Up-Effekt genannt. Dieser Effekt tritt signifikant nur bei verhältnismäßig hohen Zählraten auf. Moderne SDD Sofware bietet bereits die Möglichkeit, die Ausleserate an die Zählrate anzupassen. Außerdem geschieht das Auslesen der Ladungsträgerwolken durch 2 Kanäle: 1 Kanal mit sehr kurzer Auslesezeit, dafür schlechter Energieauflösung und ein Kanal mit langsamer Auslesezeit und besserer Energieauflösung. Wird im schnellen Kanal nur 1 Count registriert, so wird die im langsamen Kanal detektierte Energie ausgewertet. Werden im 1. Kanal mehrere Counts registiert, so wird das Ergebnis aus Kanal 2 verworfen und die Counts tragen zur Berechnung der Totzeit mit bei. Treffen jedoch 2 Photonen in so kurzem Abstand voneinander ein, dass der 1. Kanal diese als 1 Count zusammenaddiert, spricht man von einem Pile-Up-Effekt. Glücklicherweise lassen sich bei Röntgenfluoreszenzspektroskopie die falsch zugeordneten Signale auch manuell sehr leicht identifizieren denn die Energien entsprechen einer Summe aus diskreten Röntgenlinien.\newlines
Ein weiterer, regelmäßig auftretender Effekt ist die unvollständige Detektion der Ladungsträgerwolken durch Verlust von Elektron-Loch-Paaren auf dem Weg zur Ausleseelektronik. Dies liegt vor Allem an im Bereich der Oberfläche ausgelösten Elektronen. Die Eindringtiefe niederenergetischer Photonen in die Detektorfläche ist sehr gering. Demzufolge wird ein größerer Anteil der ausgelösten Ladungsträger nahe an der Oberfläche ausgelöst, wo sie entweder Rekombinieren oder ihre Ladung an die Elektrode abgeben können. Dieser Effekt wird durch flache Einfallswinkel verstärkt, unter welchen die Photonen einen längeren Weg durch die oberste Schicht der aktiven Fläche passieren müssen.\newlines
Wird ein Atom durch Strahlung ionisiert, so wird ein Anteil entsprechend $E_{kin} = h\nu - E_{bind}$ der Photonenenergie dem aus einer inneren Schale ausgelösten Elektron als kinetische Energie mitgegeben. Dieses löst durch Stoßprozesse eine Ladungsträgerwolke aus. Ein freies Elektron oder ein Elektron einer äußeren Schale wird den freigewordenen Platz in der inneren Schale einnehmen. Dabei kann es die freiwerdende Energie entweder als Auger-Elektron auf ein andere Elektron übertragen oder diese abstrahlen. Das Auger-Elektron wandelt seine Energie auch in eine Ladungsträgerwolke um, das Photon kann jedoch auch aus der Oberfläche austreten. Dadurch ist die detektierte Energie genau um die Energie des emittierten Photons niedriger. Es entstehen leicht zu identifizierende Peaks im Abstand der Photonenenergie des Si-Photons. Zur Korrektur ist es ausreichend die beiden Peaks mit und ohne Escape-Peak zu summieren.\newlines
Der Vollständigkeit halber werden noch Bragg-Peaks, welche mit Einfalls- und Ausfallswinkel in eine spezifische kristalline Struktur korrellieren, genannt. Sie werden von der Bragg-Bedingung beschrieben und treten vorherrschend bei höheren Photonenenergien auf. In den in dieser Arbeit zu untersuchenden Proben sind Bragg-Peaks jedoch nicht in signifikantem Umfang zu erwarten, da biologische Proben meist keine kristalline Struktur aufweisen.