In dieser Arbeit wurde das Simulationsprogramm QUADaps vorgestellt, welches den neuartigen QUAD-Detektor aufgrund einer virtuellen Beschreibung simuliert. Dieser ist im AnImaX Röntgenmikroskop verbaut und kann einen Raumwinkel von $>\SI{1}{\steradian}$ detektieren. Verglichen mit konventionellen Aufbauten kann er dadurch eine deutlich höhere Signaldichte erreichen, jedoch genügt er den etablierten Quantifizierungsansätzen der Röntgenfluoreszenzanalyse nicht mehr, da diese nur für kleine detektierte Raumwinkel gelten. \newline

Die Untersuchungen von Simulationen von 2- und 3-Schicht System haben gezeigt, dass weitere Schritte in Richtung einer experimentellen Validierung der QUADaps Software möglich sein sollte. Dies gelingt durch Nachweis von Selbstabsorptionseffekten in den entworfenen Proben unter Benutzung verschiedener Probendesigns, Materialien und Anregungsenergien. Daran anknüpfend kann die QUADaps Software dazu genutzt werden um Quantifizierungsansätze für große Raumwinkel und inhomogene Proben zu finden. \newline

Die für die QUADaps Software nötigen Modellannahmen wurden erläutert und die experimentellen Probenparameter anhand der Detektorgeometrie hergeleitet. Außerdem wurde gezeigt, dass der Entwurf von Proben, welche nicht symmetrisch bezüglich des softwareinternen Koordinatensystems sind, möglich ist. \newline

Abgesehen von der experimentellen Untersuchung der hier präsentierten Proben ist als nächster Schritt ist eine systematische Untersuchung von Simulationen kleiner Inhomogenitäten in homogenen Proben sinnvoll um mehr über deren Einfluss in realen Messungen zu erfahren. Langfristig können diese Erkenntnisse genutzt werden um bei Untersuchungen an biomedizinischen Proben Näherungen für eine Quantifizierung zu erarbeiten und diese anschließend zu verifizieren.